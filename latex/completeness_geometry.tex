\documentclass[12pt]{article}

% Language setting
% Replace `english' with e.g. `spanish' to change the document language
\usepackage[english]{babel}

% Set page size and margins
% Replace `letterpaper' with`a4paper' for UK/EU standard size
\usepackage[letterpaper,top=2cm,bottom=2cm,left=3cm,right=3cm,marginparwidth=1.75cm]{geometry}

% Useful packages:
\usepackage{braket}
\usepackage{amsmath}
\usepackage{graphicx}
\usepackage[colorlinks=true, allcolors=blue]{hyperref}
%\usepackage{unicode-math}
\usepackage{mathtools}
\usepackage{amsfonts}
\usepackage{booktabs}
\usepackage{caption} 
\captionsetup[table]{skip=5pt}
\usepackage{calc}
\usepackage[section]{placeins}
%\usepackage{minted}
\usepackage{accents}
\usepackage{float}
% set parameters:
\newcommand{\ubar}[1]{\underaccent{\bar}{#1}}
\newlength{\maxmin}
\setlength{\maxmin}{\widthof{$\max$}-\widthof{$\min$}}
\setcounter{tocdepth}{4}
\setcounter{secnumdepth}{4}

% macros:
\def\D{\displaystyle}
\def\att{                    % mark at the margin
        \marginpar[ \hspace*{\fill} \raisebox{-0.2em}{\rule{2mm}{1.2em}} ]
        {\raisebox{-0.2em}{\rule{2mm}{1.2em}} }
        }
\def\at#1{[*** \att #1 ***]}  % text needing attention
\def\spc{\hspace*{0.5cm}} 			% indentation


\begin{document}

\section{Completeness of molecular features}
Features of a molecular geometry (or configuration) is said to be complete if they describe distinct (numerical) values for each distinct molecular geometry; otherwise if the features of distinct geometries result in equal numerical value, then the features are incomplete. One way to check the completeness is by conducting a quality test of the molecular features.
Consider the lattices
\begin{equation}
    \mathcal{L} = \sqrt[]{2}Z_3, D_3,
\end{equation}
they have a fixed set of distances
\begin{equation}
    \mathcal{D} := \{0, \sqrt{2}, \sqrt{4}, \sqrt{6}, \sqrt{8}, \sqrt{10}, ...\}.
\end{equation}
Let $x = (x_1, x_2, x_3)$ be the 3D-cartesian coordinates, and let
\begin{equation}
    S := \{x \mid \left\| x\right\|_2 < 3 \},
\end{equation}
which implies the feasible set of distances
\begin{equation}
    \mathcal{D_F} := \{0, \sqrt{2}, \sqrt{4}, \sqrt{6}, \sqrt{8}\}.
\end{equation}

\end{document}



    
